
\section*{3. Proposed Solution}

In this section you will explain how solve the problem, that is, how
you intend to carry the project out. At this early stage you need to
be both clear about what you are going to do and flexible enough to
adapt to changing circumstances. Making an early plan will not prevent
you from running into trouble, but it will help you identify possible
problems early. For example, if you intended to run an experiment in
HCI, you might realise early on that there would be problems gathering
sufficient data to get reliable results, and that you should re-design
your experiment.

Part of the planning process involves producing a timetable for when
the work is actually going to be done.

Each part of the project should produce some output. For example you
might plan on spending two weeks on background reading: the output of
this will be a bibliography, and a possibly a literature survey for
your report. Indeed, if you take the advice given above about having
specific, measurable goals, you should describe this part of your
project as:

\begin{itemize}
\item[\bf Good] Produce bibliography (est: 2 weeks)
\end{itemize}
rather than
\begin{itemize}
\item[\bf Bad] Background reading (est: 2 weeks)
\end{itemize}

Note that the methodology you outline here is dependent upon the type
of project and engineering area. You must talk to your supervisor
about this.

\section*{4. Evaluating your Solution}

In this section you will explain how you will evaluate your solution
once you have built it. The method of evaluation will be domain
specific. Your supervisor should provide guidance as to what is an
appropriate form of evaluation. For example, user testing for a HCI
project or performance measurement for a Network Engineering project.

\section*{5. Resource Requirements}

In this section you will detail any resource requirements such as
hardware, software or access to subjects.